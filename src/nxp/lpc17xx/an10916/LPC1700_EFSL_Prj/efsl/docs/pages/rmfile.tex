\subsubsection*{Purpose}
Deletes a file.
\subsubsection*{Prototype}
\code{esint16 rmfile(FileSystem *fs,euint8* filename);}
\subsubsection*{Arguments}
Objects passed to \code{rmfile}:
\begin{itemize}
	\item{\code{fs}: pointer to the FileSystem object}
	\item{\code{filename}: pointer to the path + name of the file to be removed}
\end{itemize}
\subsubsection*{Return value}
Returns 0 if no errors are detected.\\
\newline
Returns non-zero if an error is detected, most likely that the file does not exist.
\subsubsection*{Note}
If you have opened a file with \code{fopen()}, and you wish to delete it, first
close all instances of that file. If you do not, you may corrupt the filesystem.
\subsubsection*{Example}
\lstset{numbers=left, stepnumber=1, numberstyle=\small, numbersep=5pt, tabsize=4}
\begin{lstlisting}
	#include "efs.h"

	void main(void)
	{
		EmbeddedFileSystem efsl;

		/* Initialize efsl */
		DBG((TXT("Will init efsl now.\n")));
		if(efs_init(&efsl,"/dev/sda")!=0){
			DBG((TXT("Could not initialize filesystem (err \%d).\n"),ret));
			exit(-1);
		}
		DBG((TXT("Filesystem correctly initialized.\n")));

		/* Delete some files */
		rmfile(&efsl.myFs,"file0.txt");
		rmfile(&efsl.myFs,"dir0/file0.txt");
		
		/* Close filesystem */
		fs_umount(&efsl.myFs);
	}
\end{lstlisting}
